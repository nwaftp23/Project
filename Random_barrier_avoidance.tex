\documentclass[12pt]{article}

\usepackage{amsfonts}
\usepackage{amssymb}
\usepackage{amsmath}
\usepackage{amsthm}
\usepackage{amscd}
\usepackage[letterpaper,left=2.54cm,top=2.54cm,bottom=2.54cm,right=2.54cm]{geometry}
\usepackage{graphicx, epstopdf, tikz, array}
\usepackage{tabu}
\usepackage{indentfirst}
\usepackage{subfig}
\usepackage[round]{natbib}   % omit 'round' option if you prefer square brackets
\usepackage[english]{babel}



\title{Random Barrier Avoidance with Risk-Constrained MDPs}
\author{Lucas Berry}
\date{April 21$^{\text{st}}$ 2017}
\linespread{1.5}

\begin{document}
	\maketitle
	\section{Introduction}
	When making decisions one is often confronted with the notion of risk. When choosing whether or not to go to a party that your ex might attend, to sneak out of the house when your mom might catch you or to speed when you are late for work are all examples that we face that involve a potential risky outcome. Catastrophic events have high cost and usually occur seldomly, making them hard to capture in one's model. Researchers have often tried to develop models or policies that incorporate some penalization for riskiness and these results have applications, i.e. financial mathematics, medical decision making and robotics. Each one of these fields can failure comes at a high cost, bankruptcy or economic collapse, death and broken robots.   
	
	This notion of risk-averse decision making has invaded many fields of research.  
	
	\section{Problem Formulation}
	\section{Algorithms}
	\section{Results}
	\section{Conclusion}  
	
\end{document}